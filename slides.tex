\documentclass[10pt]{beamer}

\usetheme[progressbar=frametitle]{metropolis}
\usepackage{appendixnumberbeamer}

\usepackage{booktabs}
\usepackage[scale=2]{ccicons}

\usepackage{pgfplots}
\usepgfplotslibrary{dateplot}

\usepackage{hyperref}
\hypersetup{
    colorlinks=true,
    linkcolor=blue,
    filecolor=magenta,      
    urlcolor=cyan,
}

\usepackage{xspace}
\newcommand{\themename}{\textbf{\textsc{metropolis}}\xspace}

\title{AddressSanitizer and Friends}
\subtitle{With some bonus ramblings about property based testing}
\date{\today}
\date{}
\author{Louis Taylor}
%\institute{Jetstack}
% \titlegraphic{\hfill\includegraphics[height=1.5cm]{logo.pdf}}

\begin{document}

\maketitle

\begin{frame}{Table of contents}
  \setbeamertemplate{section in toc}[sections numbered]
  \tableofcontents %[hideallsubsections]
\end{frame}

\section{Introduction}

\begin{frame}[fragile]{Introduction}

Danger level: high

\begin{itemize}
    \item undefined behaviour
    \item buffer overflows
    \item memory leaks
    \item uninitialised memory
    \item integer overflows
\end{itemize}
\end{frame}

\section{Sanitizers}

\begin{frame}[fragile]{Sanitizers}
A set of tools developed by Google,

Supported by GCC (from 4.8) and Clang (from 3.1)

Gives feedback as soon as a problem is found
\end{frame}

\section{-fsanitize=address}

\begin{frame}[fragile]{-fsanitize=address}
  Easy to get started:
  
  \begin{verbatim}gcc -fsanitize=address -o demo1 demo1.c\end{verbatim}
  
  Use binary as normal

\end{frame}

\begin{frame}[fragile]{-fsanitize=address}
  Demo:

  \begin{verbatim}demo1.c
demo2.c\end{verbatim}

\end{frame}

\section{-fsanitize=undefined}

\begin{frame}[fragile]{-fsanitize=undefined}
  Detects undefined behaviour at runtime

  \begin{verbatim}-fsanitize=alignment
-fsanitize=bool
-fsanitize=builtin
-fsanitize=bounds
-fsanitize=enum
-fsanitize=float-cast-overflow
-fsanitize=float-divide-by-zero
-fsanitize=function
-fsanitize=implicit-unsigned-integer-truncation
-fsanitize=implicit-signed-integer-truncation
-fsanitize=implicit-integer-sign-change
-fsanitize=integer-divide-by-zero
-fsanitize=nonnull-attribute
\end{verbatim}

\end{frame}

\begin{frame}[fragile]{-fsanitize=undefined}

  \begin{verbatim}-fsanitize=null
-fsanitize=nullability-arg
-fsanitize=nullability-assign
-fsanitize=nullability-return
-fsanitize=object-size
-fsanitize=pointer-overflow
-fsanitize=return
-fsanitize=returns-nonnull-attribute
-fsanitize=shift
-fsanitize=signed-integer-overflow
-fsanitize=unreachable
-fsanitize=unsigned-integer-overflow
-fsanitize=vla-bound
-fsanitize=vptr\end{verbatim}

\end{frame}

\section{-fsanitize=thread}

\section{Property based testing}

\begin{frame}[fragile]{Property based testing}
 
 ``The thing that QuickCheck does''
 
 Define a property, and ensure the property holds true for any given input
 
 Shrink the input to find a minimal failing case
 
 Like fuzzing, except fuzzing is generally trying to find crashes
 
 \url{https://github.com/silentbicycle/theft}

\end{frame}

\begin{frame}[fragile]{Property based testing}
 
 Property based testing + sanitizers
 
 Sanitizers only work with enough coverage
 
 Powerful combination
 
\end{frame}

\section{Summary}

\begin{frame}[fragile]{Summary}
 Sanitizers are good, use them if you can
 
 Property based testing is good, use it if you can
 
 Using them together is good, try it if you can

\end{frame}


\begin{frame}{Questions?}

  Get the demos and some other bits from:

  \begin{center}\url{github.com/kragniz/sanitize_things}\end{center}

\end{frame}

\end{document}
